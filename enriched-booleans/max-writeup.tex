% There are different document classes for different types
% of structures, e.g. books, articles, and letters.
\documentclass{article}

\usepackage{graphicx}
\usepackage{enumerate}
\usepackage{fancyhdr}
\usepackage{listings}
\usepackage{amsmath}
\usepackage{amssymb}
\setlength{\headheight}{15pt}
\pagestyle{fancyplain}
\usepackage{fancyhdr}
\usepackage{lastpage}

\setlength{\parindent}{0pt}
\setlength{\parskip}{2ex}

\lhead{\fancyplain{}{Enriched Booleans}}
\chead{Max Goldstein}
\rhead{COMP 150FP, Fall 2013}
\lfoot{}
\cfoot{}
\rfoot{}

\begin{document}
The syntactic proof system contains two rules, each corresponding to an instance
declaration. First, however, the arity of the functions is all that is expressed
in the type class header:
\begin{verbatim}
import Prelude hiding (either)

class Enriched a where
    opposite :: a -> a
    either :: a -> a -> a
    both :: a -> a -> a
\end{verbatim}
The first rule is an axiom that any Boolean may be enriched:
\[
    \frac{}{\textrm{Bool }\epsilon\textrm{ Enriched}}
\]
The corresponding instance declaration gives the operations that perform as
desired on Booleans; that is, the result of an Enriched once further
applied:
\begin{verbatim}
instance Enriched Bool where
    opposite = not
    either = (||)
    both = (&&)
\end{verbatim}
The second rule of the syntactic proof states that a function from an Enriched
to an Enriched is itself an Enriched. Inductively, this means that these functions
may take in Enriched functions instead of Booleans, and this propagates as many
layers as needed.
\[
    \frac{b\:\epsilon\textrm{ Enriched}}{(a \rightarrow b)\:\epsilon\textrm{ Enriched}}
\]
The first line of the instance declaration is essentially the syntactic proof
rule written in code. The implementations of the Enriched functions are hidden
in plain sight: evaluate the enriched function and then use that as an argument
to the ``same" Enriched function as was called. This ``same" function may be a
recursive call, if the argument function is still not fully applied, or it may
be a call to the base case in the other instance.
\begin{verbatim}
instance Enriched b => Enriched (a -> b) where
    opposite f x = opposite (f x)
    either f g x = either (f x) (g x)
    both f g x = both (f x) (g x)
\end{verbatim}
\end{document}

